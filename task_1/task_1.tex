\documentclass{article}
\usepackage[utf8x]{inputenc}
\usepackage[english,russian]{babel}
\usepackage{cmap}
\usepackage{amsthm,amssymb}
\usepackage[left=2cm,right=2cm,top=2cm,bottom=2cm,bindingoffset=0cm]{geometry}

\newcommand{\tasktitle}[1] {
	\begin{center}
		{\large\bf {#1}}
	\end{center}
}


\newenvironment{task}[1] {
	\noindent\fbox{\bf {#1}}
}

\begin{document}
	
	\tasktitle{Задание 1.}
	
	\begin{task}{Прогрессивная последовательность, прогрессивная ли?
		}
	\end{task}
	\begin{proof}[Условие]
		Из множества $\{1,2,\cdots,97\}$ выбирают три числа. Какова вероятность, что из них можно составить арифметическую прогрессию?
	\end{proof}
	\begin{proof}[Решение]
		Прежде всего, заметим что в множестве $\{1,2,\cdots,97\}$ всего:
		\begin{enumerate}
			\item 49 нечетных чисел: $\{1,3,\cdots,95,97\}$
			\item 48 четных чисел: $\{2,4,\cdots,94,96\}$
		\end{enumerate}
		
		Далее, для удобства, будем считать, что арифметическая прогрессия -- возрастающая арифметическая прогрессия, поскольку любое множество чисел, составляющее убывающую арифметическую прогрессию, -- $a_1, \cdots, a_n$, можно представить в виде убывающей арифметической прогрессии, -- $a_n, \cdots, a_1$. 
		
		Любая ройка чисел ${a_1,a_2,a_3}$, образующая арифметическую (возрастающую) прогрессию имеет вид: $a_1, a_2 = a_1 + d, a_3 = a_1 + 2d$, где $d$ -- разность арифметической прогрессии.
		
		Заметим, что числа $a_1$ и $a_3 = a_1 + 2d$ имеют одинаковую четность, поскольку добавление четного числа ($2d$) не меняет четности. Также, заметим, что любая пара чисел $(a_1, a_1 + 2d)$ однозначно определяет $a_2$, последнее из тройки чисел, образующих арифметическую прогрессию: 
		$$a_2 = \frac{a_1 + a_3}{2} = \frac{a_1 + a_1 + 2d}{2} = a_1 + d$$
		
		Тогда, поскольку два числа арифметической прогрессии однозначно определяют третье число, то количество троек, из которых можно составить арифметическую прогрессию, можно найти следующим образом: это сумма количества всевозможных способов выбрать 2 нечетных числа из всех нечетных чисел и количества всевозможных способов выбрать 2 четных числа из всех четных чисел.
		
		Нечетных чисел всего 49, поэтому количество всевозможных способов выбрать 2 нечетных числа из всех нечетных чисел, которых всего 49, равно $C_{49}^2$. Аналогично, четных чисел всего 48, поэтому количество всевозможных способов выбрать 2 четных числа из всех четных чисел, которых всего 48, равно $C_{48}^2$.
		$$C_{48}^2 + C_{49}^2 = \frac{48!}{2!46!} + \frac{49!}{2!47!} = \frac{48 \cdot 47}{2} + \frac{49 \cdot 48}{2} = \frac{48 \cdot 47 + 49 \cdot 48}{2} = \frac{96 \cdot 48}{2} = 48^2 = 2304$$
		
		Количество благоприятных исходов в искомой вероятности $p$ равно 2304, а количество всевозможных исходов, равно количеству всевозможных способов выбрать 3 числа из данных нам 97-ми, что равно $C_{97}^3$.
		Тогда искомая вероятность $p$ равна:
		$$p = \frac{C_{48}^2 + C_{49}^2}{C_{97}^3} = \frac{2304}{\frac{97 \cdot 96 \cdot 95}{3 \cdot 2}} = \frac{2304}{97 \cdot 16 \cdot 95} = \frac{2304}{147440} = \frac{144}{9215}$$
		
		Таким образом, вероятность, что из трех выбранных чисел множества $\{1,2,\cdots,97\}$ можно составить арифметическую прогрессию, равна $\frac{144}{9215}$.
\end{proof}

	
\end{document}