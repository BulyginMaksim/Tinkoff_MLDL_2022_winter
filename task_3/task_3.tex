\documentclass{article}
\usepackage[utf8x]{inputenc}
\usepackage[english,russian]{babel}
\usepackage{cmap}
\usepackage{amsthm,amssymb}
\usepackage[left=2cm,right=2cm,top=2cm,bottom=2cm,bindingoffset=0cm]{geometry}

\newcommand{\tasktitle}[1] {
	\begin{center}
		{\large\bf {#1}}
	\end{center}
}


\newenvironment{task}[1] {
	\noindent\fbox{\bf {#1}}
}

\begin{document}
	
	\tasktitle{Задание 3.}
	
		\begin{task}{Three boys, one course}
		\end{task}
	\begin{proof}[Условие]
		Трое ребят Вася, Петя и Миша поступают на курсы по машинному обучению. Известно, что из них троих возьмут только двоих, причем Вася следил за преподавателями и узнал, что Петю точно берут. Какова вероятность того, что Васю тоже возьмут?
	\end{proof}
	\begin{proof}[Решение]
		Поскольку мальчиков всего трое, а на курс возьмут только двоих, то существует всего $C_3^2 = 3$ возможных пар выбора двух мальчиков из трех:
		\begin{enumerate}
			\item Петя и Миша
			\item Петя и Вася
			\item Вася и Миша
		\end{enumerate}
	
	Поскольку Петю точно берут, то исход, когда на курс берут пару мальчиков из Васи и Миши, невозможен. 
	Поэтому, остается два возможных исхода того, какую пару возьмут на курс -- либо Петя и Миша, либо Петя и Вася. 
	
	По условию задания, нас просят найти вероятность того, что Васю тоже возьмут: благоприятных исходов всего одно -- это пара из Пети и Васи, а всего возможных исхода два. Тогда, вероятность $p$ того, что Васю тоже возьмут, равна:
	$$p = \frac{\#\textit{исходы, где есть Вася}}{\#\textit{все исходы}} = \frac{1}{2} = 0.5$$
	
	Таким образом,  вероятность того, что Васю тоже возьмут, равна $0.5$.
\end{proof}
	
\end{document}