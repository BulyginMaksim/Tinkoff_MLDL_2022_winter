\documentclass{article}
\usepackage[utf8x]{inputenc}
\usepackage[english,russian]{babel}
\usepackage{cmap}
\usepackage{amsthm,amssymb}
\usepackage[left=2cm,right=2cm,top=2cm,bottom=2cm,bindingoffset=0cm]{geometry}

\newcommand{\tasktitle}[1] {
	\begin{center}
		{\large\bf {#1}}
	\end{center}
}


\newenvironment{task}[1] {
	\noindent\fbox{\bf {#1}}
}

\begin{document}
	
	\tasktitle{Задание 5.}
	
		\begin{task}{Друзья останутся без подарков?}
		\end{task}
	\begin{proof}[Условие]
		Известно, что $n$ друзей собрались и решили отправить письма Деду Морозу. У каждого есть 11 конверт и 11 письмо. Потом ребята всё перемешали и стали класть письма в конверты. Сколько в среднем писем попадёт в свой конверт?
	\end{proof}
	\begin{proof}[Решение]
		Введем $\xi$ -- случайная величина количества писем, которые попали в свой конверт после перемешивания.
		
		Дополнительно, введем $\xi_i$, где $i \in \{1, \cdots, n\}$ -- случайная величина, которая принимает значение 1, если в $i$-ый
		конверт попало свое письмо, и 0, если в $i$-ый конверт попало не свое письмо.
		
		Очевидно, что $\xi = \xi_1 + \cdots + \xi_n$. Тогда найдем распределение случайных величин $\xi_i \;\; \forall i \in \{1, \cdots, n\}$:
		понятно, что нужное письмо попадет в $i$-ый конверт с вероятностью  $p_1 = \frac{1}{n}$, поскольку из $n$ писем нужное
		всего одно, также, понятно что нужное письмо не попадет в $i$-ый конверт с вероятностью  $p_2 = \frac{n - 1}{n}$,
		поскольку из $n$ писем не подходящих в $i$-ый конверт писем всего $n-1$ штука. Тогда $\forall i \in \{1, \cdots, n\}$ случайная величина $\xi_i$ имеет распределение:
		
		\begin{table}[h]
			\begin{center}
				\begin{tabular}{ |c|c|c|}
					\hline
					$\xi_i$ & 0 & 1 \\  
					\hline
					$p$ & $\frac{n-1}{n}$ & $\frac{1}{n}$\\
					\hline
				\end{tabular}
			\end{center}
		\end{table}
	
	Теперь $\forall i \in \{1, \cdots, n\}$ мы можем вычислить математическое ожидание случайной величины $\xi_i$:
	$${\mathbb{E}}\xi_i = 0 \cdot \frac{n-1}{n} + 1 \cdot \frac{1}{n} = \frac{1}{n}$$
	
	Теперь, воспользуемся тем, что $\xi_i$ одинаково распределены $\forall i \in \{1, \cdots, n\}$, и тем, что матожидание суммы равно сумме матожиданий и найдем  ${\mathbb{E}}\xi$:
	$${\mathbb{E}}\xi = {\mathbb{E}} \left(\xi_1 + \cdots + \xi_n \right) = {\mathbb{E}}\xi_1 + \cdots + {\mathbb{E}} \xi_n = 
	n \cdot {\mathbb{E}} \xi_i = n \cdot \frac{1}{n}  = 1$$
	
	Таким образом, поскольку ${\mathbb{E}}\xi  = 1$, то в среднем ровно одно письмо попадет в свой конверт.
\end{proof}
	
\end{document}